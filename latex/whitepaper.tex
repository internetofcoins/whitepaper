\documentclass[a4paper]{article}

\title{Internet of Coins\\[6mm]
\small{Hybrid Assets for Peer-to-Peer Inter-Blockchain Value Transfer}
}

\makeindex

\begin{document}
\maketitle

\begin{abstract}
 A meta-protocol and blockchain integration solution for transacting value across different digital currency systems would allow for multiple decentralized financial platforms to exchange value and thus form a coherent cryptosphere, without needing intermediary financial institutions. Third party services currently assist users to exchange one form of digital cash or asset for another, but a trusted third party is still required to mediate the transactions. We propose a solution to the problem of isolated digital currency systems using a meta-level transfer protocol with an extendable design, making accessible any kind of blockchain-based economy or other digital cash system for cross-blockchain and inter-system transactions. A dynamic proof-of-allocation mechanism provides verification of solvency to any of the transferable assets on such a high-level platform, and makes it possible for anonymous allocation providers to earn rewards to be part of an autonomous, decentralized exchange system. Simultaneously, any network participant may act as an agent for the realization of an encrypted exchange of value between two peers. Information is transacted across the blockchains or digicash systems of the respective value-holders being traded, while node connections are directly negotiated without retaining or storing any records. As with Bitcoin, the network itself requires minimal structure, and messages are broadcast on a best effort basis. Peer connections employ failover restructuring of transactions and messaging, and nodes can leave and rejoin the network at will, updating their allocation tables from any random node.
\end{abstract}

\section{Introduction}
Since the inception of Bitcoin, we have seen the rise of many digital cash descendants that either imitate or replicate the peer-to-peer version of electronic cash that Bitcoin was designed to be. Next to blockchain technology, decentralized asset exchangesi have been built to facilitate the issuance of virtual assets. Ledgerless \cite{opentransactions} systems are also in development. While the existing solutions may function as intended, exchanges in value between these systems and blockchains is most often done through trusted third parties. This re-introduces the inherent weaknesses of the traditional trust based financial models into the cryptosphere of peer-to-peer digital currencies. In many cases this is no direct threat to the usage and trade of these currencies. Yet at the same time it isolates the different implementations of digital cash from eachother, except for their value being tied to Bitcoin. This exposes these currencies to price manipulation strategies that have the potential to drain all value from these markets. \cite{panture}


What is needed is a hybrid asset and transfer system utilizing a modular and standardized peer-to-peer platform, instead of trust. A loosely coupled systemiv allowing any two willing parties to transact assets cross-blockchain with each other, without the need for a trusted third party. Transaction methods that have no systemic boundaries and can operate cross-blockchain tie different value systems together. Through hybrid assets, loosely coupled financial systems make the cryptosphere immune to intermediary control by trusted parties and stabilizes crypto-economies by insuring the value of smaller smaller markets can transfer into hybrid assets in the event of network destabilization. In this paper, we propose a solution to the digital economy diaspora using a peer-to-peer, distributed, inter-blockchain exchange server to mediate the transfer of value between digital value systems.

\section{Inter-Blockchain Transactions}
\section{Meta Server}
\section{Proof-of-Allocation}
\section{Network}
\section{Incentive}
\section{Unified Asset Descriptor}
\section{Cross-Blockchain Distributed Autonomous Organizations}
\section{Value and Allocation Theory}
\section{Haystack Transactions}
\section{Calculations}
\section{Conclusion}



\section{References}

\bibliography{internetofcoins}{}
\bibliographystyle{plain}

\end{document}
